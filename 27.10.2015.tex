% !TeX spellcheck = uk_UA
\subsection{Набуття прав}
Права на об'єкти АП виникають автоматично за фактом завершення (правильніше - остаточного припинення) роботи над твором.
Для виникнення і здійснення прав АП не потрібно жодних обов'язкових формальностей. АП поширюється на (не)оприлюднені і (не)завершені твори. 

В будь-який час протягом терміну охорони, АП його власник або довірена особа можуть подати заяву про реєстрацію АП і отримати відповідне свідоцтво. 

В АП діє презумпція авторства - особа вважається автором, поки не доведено протилежне. У зв'язку з цим свідоцтво про реєстрацію АП не є повноцінним охоронним документом і воно може бути оскаржене і анульоване. 

Також - головний баг - АП охороняє твір лише у формі його виконання, тобто ідеї і принципи, втілені у творі, не охороняються.

Для сповіщення про свої права власник АП може використовувати спеціальне позначення, що розміщується на оригіналі та всіх копіях \textcopyright найменування автора, рік першого правомірного оприлюднення. Якщо особа автора достаменно невідома, то вважається, що особа, яка вперше оприлюднила твір, представляє інтереси автора.

\paragraph{Особливості охорони комп’ютерних програм}
Хоча формально є об’єктом АП, повинна бути захищена комплексно.
\begin{itemize}
	\item програма як така захищається як об’єкт авторського права.
	\item Алгоритм захищається патентним правом як спосіб обробки інформації - в міжнародній патентній класифікації розділ G;
	\item інтерфейс захищається як промисловий зразок. (мкпз 14-04 Комп’ютерні інтерфейси)
\end{itemize}
% а если сделать исследование по поводу лизенций opensource и украинского законодательства, на 6-м курсе не будет самостоятельных работ
\subsection{Співавторство}
Якщо твір створений кількома фізичними особами то можна говорити про співавторство при виконанні наступних 4-х умов:

/*чего-то про патологоанатомов - Юрген Торвальд "Век криминалистики"(lite) "100 лет криминалистики"(full), "Следы в пыли"*/

\begin{itemize}
	\item спільна творча праця
	\item цілісність твору (при вилученні внеску одного з авторів - твір втрачає цілісність)
	\item добровільність
	\item наявність домовленості про спільну творчу працю.
\end{itemize}

Співавторство - роздільне і нероздільне. При роздільному співавторстві внесок кожного автора є очевидним (пісня з слів та музики). При нероздільному важко або неможливо виділити внески (брати Стругацькі). При роздільному співавторстві кожен із авторів має вільне право на використання свого внеску. При нероздільному - кожен має на використання всього твору, якщо не суперечить інтересам інших. Інтерв’ю - порівну. Взагалі, авторський внесок ділиться порівну, якщо іншого не зазначено.
Якщо авторів більше 3х (внесок більше третини) - краще ухвалювати авторський договір.
При оформленні такої штуки треба слідкувати, щоб не було "загального редагування" у співавторстві. Бо тоді буде нероздільне.

\subsection{Термін дії АП}
Починає діяти з моменту остаточного припинення дії над твором. А кінцевий термін, в загальному випадку, прив’язується до смерті автора. Для зручності підрахунків прийнято, що всі фіксовані сроки відраховуються від першого січня року наступного за роком, в якому мали місце юридичні події. Особисті немайнові права оформляються безстроково. 

В загальному випадку правова охорона АП завершується через 70 років з року смерті автора (в Європи - від 50ти). Якщо твір оприлюднено анонімно або під псевдонімом, завершується через 70 років з року першого оприлюднення. Якщо особа автора не викликає сумнівів, то як в загальному випадку. Якщо твір оприлюднюється частинами, то для кожної частини терміни використовуються окремі (актуально для попереднього положення). Правова охорона твору, створеного у співавторстві, завершується через 70 років \textbf{з року (не з дати)} смерті останнього співавтора.
Правова охорона творів посмертно реабілітованих авторів завершується через 70 років після реабілітації.
Якщо твір вперше оприлюднено більше ніж через 70 років з року смерті автора, то рахується з дня оприлюднення (? неоднозначно)
Якщо твір вперше оприлюднено після терміну правової охорони, то особа, що оприлюднила твір, набуває права, подібні до авторських. Термін дії цих прав - 25 років.

\subsection{Суміжні права}
/*Свідоцтво - прив’язка реквізиту до твору.*/
Суміжні права поширюються на виконання об’єктів АП з метою донесення до кінцевого споживача. Термін СП вживається у множині тому що є ієрархія, згідно з якою кожен нижчий рівень при реалізації прав повинен враховувати інтереси всіх верхніх. Наприклад, АП належать композитору і автору слів. Виконавець повинен враховувати інтереси композитора, автора. Студія запису - автора, композитора і співака. 

Для виникнення і здійснення СП не потрібно жодних обов’язкових формальностей - виникають за фактом першого правомірного виконання твору. Термін дії - 50 років (в європах - 20 років) з року першого виконання.

Для сповіщення про свої права використовують позначення, що викоритовується (P) 