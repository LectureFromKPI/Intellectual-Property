ВКМ - вирішення утилітарної задачі в будь-якій галузі промисловості або іншій сфері суспільно корисній діяльності людині або іншій сфері, яке відповідає визначеним законодавством умовам надання правової охорони і визнане як ВКМ уповноваженим державним органом.

\subsection{Варіанти рішень стосовно правової охорони}
\begin{itemize}
	\item Зберігати відомості про ВКМ в таємниці в лише для власного використання. Цей спосіб також використовується тоді, коли можливі зловживання із правом попереднього використання з боку конкурентів.
	\item
	\item Подати заявку і отримати патент. Винахідник розкриває всі відомості стосовно винаходу для подальшого розвитку суспільства, науки і техніки, а держава надає правову охорону. 
\end{itemize} 

На даний час діє чотири види патентів:
\begin{itemize}
	\item Патент на винахід 
	\item Патент на секретний винахід
	\item Патент на корисну модель
	\item Патент на секретну корисну модель
\end{itemize}

В СРСР діяли авторські свідоцтва. До 2011 року діяли деклараційні патенти.

\subsection{Об’єкти ВКМ}
Продукт
- пристрій 
- речовина (індивідуальна сполука або композиція)
- штам мікроорганізму
- клітинна культура рослини або тварини

Спосіб або процес:
- наявність операцій
- послідовність операцій
- умови виконання операцій (режими та/або обладнання і матеріали)

Застосування відомого продукту або способу за новим призначенням.

Об’єкт винаходу може бути одиничний або комбінований, але об’єднанний спільним винахідницьким засобом.

\subsection{Не можуть бути об’єктами ВКН}
\begin{itemize}
	\item Математичні методи
	\item Умовні позначення, правила
	\item Пропозиції, які стосуються лише зовнішнього виду виробу
	\item Проекти і схеми планування
	\item Комп’ютерні програми і бази даних як такі
	\item Пропрозиції в основі яких лежать біологічні процеси відтворення
\end{itemize}

\subsection{Умови надання правової охорони}
\begin{itemize}
	\item Новизна (глобальна)
	\item Винахідницький рівень
	\item Промислова придатність
\end{itemize}
Моделі:
\begin{itemize}
	\item Новизна (локальна)
	\item Промислова придатність
\end{itemize}
Якщо вона не є частиною рівня техніки, тобто не була відома до дати подання заявки або дати встановлення пріоритету.

Пропозиція вважається промислово придатною, якщо вона реалізовується у промисловості або іншій сфері.

Вважається, що пропозиція має винахідницький рівень, якщо вона не випливає явно із рівня техніки, тобто не є очевидною для відповідного фахівця.

\subsection{Процедура отримання патенту}
Заявка подається особисто або по пошті. Але, заявки на секретні ВКМ подаються тільки особисто. Через декілька тижнів приходить розписка про отримання матеріалів, в якій заявці присвоюється індивідуальний номер. Номер має таку структуру: латинська літера, рік подання патенту. Ще через декілька тижнів приходить лист про встановлення дати пріоритету. За заявкою на КМ проводиться попередня і формальна експертиза. За заявкою на винахід проводиться попередня, формальна і кваліфікаційна експертиза. За результатами експертизи надсилається повідомлення. Для проведення кваліфікаційної експертизи потрібно додаткове клопотання. Попередня експертиза - це на предмет секретності. Формальна - на предмет відповідності об’єкта, правильності оформлення заявки і сплати зборів. І кваліфікаційна - на предмет умов отримання правової охорони. На винахід - 30 місяців. Після позитивного рішення про видачу патенту відбувається публікація відомостей про нього в офіційному бюлетені "Промислова власність".