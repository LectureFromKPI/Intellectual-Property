\section{Система інтелектуальної власності}\marginpar{\framebox{01.09.2015}}
\subsection{Основні поняття та визначення}
\textbf{Право ІВ} - це право на результат творчої діяльності людини.

\textbf{ОПІВ} - об’єкт права інтелектуальної власності. \textbf{Завжди} нематеріальний.

Право ІВ має подвійну природу: 
\begin{itemize}
	\item Особисте немайнове право
	\item Майнове право
\end{itemize}

Особисте немайнове право невіддільне від автора і немає обмежень у просторі та часі. Складається з
\begin{itemize}
	\item Право людини на визнання її творцем ОПІВ
	\item Право на перешкоджання такого використання ОПІВ, яке може завдати шкоди честі чи репутації автора
	\item Інші особисті немайнові права передбачені законодавством (право оприлюднювати твір анонімно, під псевдонімом і так далі)
\end{itemize}

Майнові права віддільні від автора і мають обмеження у просторі та часі. Складаються з
\begin{itemize}
	\item Право використовувати ОПІВ у власній господарській діяльності
	\item Виключне право дозволяти використання ОПІВ іншим особам
	\item Виключне право перешкоджати неправомірному використанню ОПІВ
	\item 
\end{itemize}
