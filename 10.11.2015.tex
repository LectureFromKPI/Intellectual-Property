% !TeX spellcheck = uk_UA
Надалі конспект буде трохи перекликатися з конспектом на http://patent.inf.ua/.

\section{Набуття прав на промислові зразки}

\textbf{Промисловий зразок} - результат творчості людини у галузі ?. По суті ПЗ являє собою зовнішній вигляд промислового виробу. Точніше, поняття ПЗ поширюється на вироби, отримані промисловим або кустарним способом (тобто із розділенням операцій) (не ремісничий). 

Об’єктом ПЗ може бути форма, малюнок, розфарбування або будь-яке їх поєднання, які визначають зовнішній вигляд виробу і призначені для задоволення естетичних та ергономічних потреб. Тобто, ПЗ не вирішує утилітарне питання як винахід або корисна модель.

Не можуть бути об’єктами ПЗ:
\begin{itemize}
	\item пропозиції, що містять тільки технічний опис (?)
	\item промислові та інші подібні споруди (греблі, естакади, мостові переходи, димові труби, гравірні, віадуки, акведуки)
	\item об’єкти архітектури (крім малих архітектурних форм (МАФи) )
	\item друковані видання як такі (але не шрифти і орнаменти)
	\item об’єкти нестійкої форми (фонтани, феєрверк)
\end{itemize}

З недавнього часу комп’ютерний інтерфейс може бути об’єктом промислового зразка (МКПЗ - 1404?)

Охоронним документом є патент. Термін дії патенту - 10 років з можливістю подовження один раз, але не більше, ніж на 5 років. Відраховується від дати подання заяви, або дати встановлення пріоритету. 

Процедура подібна до КМ (корис. моделі) - подається заява до укрпатенту. Права починають діяти з моменту публікації відомостей. 

Обсяг правової охорони ПЗ  визначається сукупністю його суттєвих ознак, зафіксованих на матеріальному носії. Умовою надання правової охорони є оригінальність. ПЗ вважається оригінальним, якщо сукупність його суттєвих ознак не була відома у Відомстві (укрпатент) до дати подання заяви або дати встановлення пріоритету. Тобто це означає, що експертиза на глобальну новизну не відбувається.

У зв’язку з цим патент носить характер деклараційного і видається на підставі презумпції авторства під відповідальність власника (тобто може бути досить легко оскаржене). 

/* Промислові зразки досить складно класифікувати. Класи - фунціональна належність. Підкласи - алфавітний порядок за франц. мовою. При визначенні класу потрібно користуватися лише останніми версіями класифікації.

Промзразок є також засобом індивідуалізації зразків. (наприклад, решітки радіаторів на автомобілях)*/

Також буває важко відрізнити ПЗ (об’єкт промислової власності) від твору ужиткового мистецтва (декоративно-прикладного мистецтва) (об’єкт авторського права). Приклад - губки.

Склад заяви:
\begin{itemize}
	\item Заява?..
	\item Комплект зображень ПЗ на визначеному законодавством матеріальному носії, які розкривають всі суттєві ознаки ();
	\item Опис ПЗ, що призначений для тлумачення і узагальнення суттєвих ознак.
	\item Документ про сплату збору на подання заявки. 
	\item додаткові матеріали (за необхідності)
\end{itemize}

в переважній більшості випадків в якості зображень використовують фотографії. Якщо об’єкт 3вимірний, то зображують в 6 ти ортогональних проекціях. Плюс можуть показуватися характерні елементи. Якщо елемент трансформується під час експлуатації, їх зображують у декількох станах. Одяг фотографують на манекені. 
Зображення може бути зняте з натури або бути комп’ютертим зображенням.
Якщо об’єкт є плоскою періодичною структурою, допускається надати примірник.

Якщо комплект зображень не розкриває всіх суттєвих ознак то надається макет виробу. 

... тут ще про варіант

Якщо ще не вистачає - дається ескіз.