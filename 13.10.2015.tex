\section{Термін дії патенту}
Базовий термін дії патенту для винаходу 20 років, для корисної моделі 10 років. Він відраховується від дати подання заявки або дати встановлення пріоритету.

Для підтримання його чинності сплачується відповідний збір, ставка збору прогресуючи, від 300 гривень до 3500 гривень.

На сплату щорічного збору виділяється період розміром 4 місяці, який прив’язується до дати подачі заявки. 

В разі несплати збору чинність патенту призупиняється, але його можна відновити. Для цього, протягом року після просрочки потрібно заплатити відповідний збір і пеню в 50\%. Якщо й після року не сплачено, то дія патенту взагалі припиняється.

Термін дії патенту на секретний винахід дорівнює терміну засекречування, але не більше базового.

Після розсекречування за клопотанням власника термін дії може буде продовжено, але в рамках базового.	

Якщо для використання винаходу (і тільки його), потрібно отримання спеціального дозволу (лікарські засоби, озброєння), то термін дії патенту може бути одноразово подовжено, але не більше ніж на 5 років.

Достроково термін дії патенту може бути припинений:
\begin{itemize}
	\item За бажанням власника
	\item За несплату або несвоєчасну сплату збору для підтримання чинності
	\item За рішенням суду, у разі невідповідності умовам
\end{itemize}

\subsection{Права, що випливають з патенту}
Права, що випливають з патенту починають діяти з дати публікації відомостей про його видачу.
Включне право дозволяти або забороняти використання ВКМ іншим особами.
Право перешкоджати діям, що порушують або створюють загрозу порушення попередніх прав.
	
\subsection{Патентування за кордоном}
Будь-яка особа має право запатентувати свій ВКМ за кордоном, але це право не виключне.

Для цього необхідно спочатку подати заявку до УкрПатенту. В разі відсутності офіційної заборони протягом визначеного терміну можна продовжити патентування в інших країнах.

Якщо є бажання запатентувати ВКМ відразу в декількох країнах, то доцільно скористатись міжнародними договорами, які спрощують цю процедуру. По перше, це договір про патентну кооперацію.

\section{Набуття прав на об’єкти авторського права та суміжні права}
Авторське право поширюється на науку, твори та мистецтва
\subsection{Об’єкти АП}
Перелік об’єктів АП дає статья 8 частина 1, закону України про авторське право та суміжні права.

\begin{enumerate}
	\item Ком’ютерна програма
	\item Твори призначені для сценічного показу та їх постанов
	\item Аудіовізуальні твори
	\item Музичні твори з текстом та без тексту
	\item Твори образотворчого мистецтва.
	\item Твори ужиткового мистецтва
	\item Твори архітектури, місцебудування і садовопаркового мистецтва
	\item Фотографічні та інші подібні твори.
	\item Похідні твори
	\item Інші твори
\end{enumerate}

Не можуть бути об’єктами АП
\begin{itemize}
	\item Розклад руху транспорту, програми телепередач та інші подібні бази даних.
	\item Умовні позначення, пробірні клейма, знаки якості і тому подібне.
	\item Державна символіка та державні нагороди.
	\item Корпоративна символіка
	\item Тексти законодавчів актів та їхні офіційні переклади
	\item Грошові знаки і подібні після введення їх в обіг
\end{itemize}